\newpage
\section{Τύποι για \textlatin{QDA, LDA}}
\label{app:app1}
Στον αλγόριθμο \textlatin{QDA} ο παράγοντας $P(X|y)$ δίνεται από τον τύπο:
$$\mathlarger{P(X|y=k)=\frac{1}{(2\pi)^\frac{n}{2}\left\lvert \sum_k\right\rvert^\frac{1}{2}}\mathlarger{e}^{-\frac{1}{2}(X-M_k)^T\sum_k^{-1}(X-M_k)}}$$
όπου:
\begin{description}
    \item[$k$] η κλάση που εξετάζουμε
    \item[$X$] το διάνυσμα χαρακτηριστικών του δείγματος
    \item[$M_k$] το κέντρο των σημείων της κλάσης $k$
    \item[$n$] η διάσταση του $X$
    \item[$\sum_k$] ο πίνακας συνδιακύμανσης
\end{description}
και
$$\mathlarger{X=\left\langle x_1,x_2,\dots,x_n\right\rangle}$$
αν τα σημεία που ανήκουν στην κλάση $k$ είναι $K_1,K_2,\dots,K_m$ με:
$$\mathlarger{K_i=\left\langle k_{i_1},k_{i_2},\dots,k_{i_n}\right\rangle}$$
τότε το $M$ για αυτή την κλάση θα είναι:
$$\mathlarger{\mathlarger{M=\left\langle m_1,m_2,\dots,m_n\right\rangle =\left\langle \frac{1}{m}\sum\limits_{i=1}^{m}k_{i_1},\frac{1}{m}\sum\limits_{i=1}^{m}k_{i_2},\dots,\frac{1}{m}\sum\limits_{i=1}^{m}k_{i_n}\right\rangle}}$$
Η διακύμανση είναι μας δείχνει τη διασπορά των σημείων της κλάσης από το κέντρο
της κλάσης για μια συγκεκριμένη διάσταση και υπολογίζεται για κάθε διάσταση
ξεχωριστά. Η διακύμανση για μια διάσταση $u$ από τις $n$ διαστάσεις θα είναι:
$$\mathlarger{V_u=\frac{1}{m}\sum\limits_{i=1}^{m}\left(k_{i_u}-m_u\right)^2}$$
Πολύ παρόμοια είναι και η συνδιακύμανση η οποία όμως υπολογίζεται για δύο
διαστάσεις $u$ και $z$:
$$\mathlarger{C_{uz}=\frac{1}{m}\sum\limits_{i=1}^{m}\left(k_{i_u}-m_u\right)\left(k_{i_z}-m_z\right)}$$
Ο πίνακας συνδιακύμανσης είναι ένας συμμετρικός πίνακας με διαστάσεις
$n\times n$ και είναι:
$$
\sum =
\begin{bmatrix}
    V_1 & C_{12} & \dots & C_{1n} \\
    C_{12} & V_2 & \dots & C_{2n} \\
    \vdots & \vdots & \ddots & \vdots \\
    C_{1n} & C_{2n} & \dots & V_n
\end{bmatrix}
$$

Ο αλγόριθμος \textlatin{LDA} είναι μια απλοποίηση του \textlatin{QDA} όπου
θεωρούμε ότι όλες οι κλάσεις έχουν τον ίδιο πίνακα συνδιακύμανσης το οποίο
απλοποιεί πάρα πολύ τις πράξεις.