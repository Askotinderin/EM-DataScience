\begin{abstract}
Η συγκεκριμένη εργασία γίνεται με σκοπό την εξοικείωση μας με την επιστήμη των δεδομένων και συγκεκριμένα τις τεχνικές, μεθόδους και τους αλγορίθμους αυτής και της μηχανικής μάθησης. Στα πλαίσια της εργασίας θα αναλυθούν τα διάφορα
προβλήματα που προσπαθεί να λύσει η επιστήμη των δεδομένων όπως είναι η ταξινόμηση, η παλινδρόμηση και η ομαδοποίηση.

Η ταξινόμηση είναι και το πιο γνωστό πρόβλημα όταν κάποιος σκέφτεται την επιστήμη των δεδομένων. Θα δούμε διάφορα ήδη ταξινόμησης όπως είναι η δυαδική και η κατηγορική. Θα αναλύσουμε επίσης διάφορους από τους πιο διάσημους αλγόριθμους
ταξινόμησης, κάποιοι από τους οποίους μπορεί να είναι παραμετρικοί ενώ άλλοι όχι, και γενικότερα θα δούμε τις χρήσεις και τα πλεονεκτήματα του κάθε αλγορίθμου στον τομέα αυτόν.

Η παλινδρόμηση είναι επίσης ένα σημαντικό πρόβλημα που έρχεται να λύσει η επιστήμη των δεδομένων το οποίο είναι πολύ παρόμοιο με την ταξινόμηση. Μάλιστα υπάρχουν πάρα πολλοί αλγόριθμοι που είναι κατάλληλοι και για τα δύο προβλήματα.
Η διαφορά της ταξινόμησης με την παλινδρόμηση είναι ότι ενώ στην ταξινόμηση προσπαθούμε να κατηγοριοποιήσουμε μία είσοδο σε κάποιες διακριτές κατηγορίες και να της δώσουμε μία \tqt{ταμπέλα}, στην παλινδρόμηση προσπαθούμε να αντιστοιχίσουμε μια
είσοδο σε μία διακριτή ή και συνεχή έξοδο. Το πιο απλό παράδειγμα θα ήταν η προσέγγιση μια συνάρτησης χρησιμοποιώντας μόνο τις εισόδους και εξόδους αυτής.

Η ομαδοποίηση είναι μια μέθοδος η οποία ίσως δεν έχει τόσες χρήσεις σε σύγκριση με τις δύο προηγούμενες. Παρ\texttt{"} όλα αυτά είναι και αυτή μια εξίσου σημαντική μέθοδος καθώς και είναι μία μη παραμετρική μέθοδος (κάτι το οποίο είναι χρήσιμο στη μηχανική μάθηση). Αυτό που προσπαθεί να πετύχει είναι να χωρίσει ένα σύνολο δεδομένων σε ομάδες ανάλογα με την ομοιότητα των στοιχείων. Παρ\texttt{"} όλα αυτά δεν μπορούμε να ελέγξουμε αυτές τις ομάδες αλλά δημιουργούνται αποκλειστικά από τον
αλγόριθμο.

Επίσης θα αναλύσουμε σε βάθος την κυρίαρχη δομή που μπορεί να λύσει και τα τρία παραπάνω προβλήματα με διάφορες παραλλαγές της και αυτή είναι το νευρωνικό δίκτυο. Θα δούμε αναλυτικά τον τρόπο λειτουργίας τους και τι τα κάνει να ξεχωρίσουν
από τις υπόλοιπες τεχνικές. Επιπλέον στα πλαίσια αυτης της εργασίας θα υλοποιήσουμε το δικό μας νευρωνικό δίκτυο από την αρχή.
\end{abstract}