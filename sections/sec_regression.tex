\newpage
\section{Παλινδρόμησή}

\subsection{Γραμμική Παλινδρόμηση}
Η ποιο απλή και ευρέως χρησιμοποιούμενη τεχνική παλινδρόμησης είναι η γραμμική. Όπως
υποδεικνύει το όνομα της η γραμμική παλινδρόμηση συσχετίζει μία ανεξάρτητη με μία
εξαρτημένη μεταβλητή με μία γραμμική σχέση προσαρμόζοντας μία ευθεία γραμμή
παλινδρόμησης πάνω στα δεδομένα. Η ευθεία δίνεται από την σχέση:
$$y=ax+b+c$$
όπου:
\begin{description}
    \item[$y$ :] εξαρτημένη μεταβλητή (\textlatin{dependent variabl}e)
    \item[$x$ :] ανεξάρτητη μεταβλητή (\textlatin{Independent / explanatory variable or regressor})
    \item[$a$ :] κλίση της ευθείας (\textlatin{slope})
    \item[$b$ :] σημείο τομής με τον άξονα $y$ ($y$ – \textlatin{intercept})
    \item[$c$ :] σφάλμα παλινδρόμησης (\textlatin{regression residual / error})
\end{description}

Η ευθεία αυτή προσαρμόζεται στα δεδομένα με την χρήση της τεχνικής του μικρότερου
αθροίσματος των τετραγώνων (των \textlatin{residuals}). Τα \textlatin{residuals}(σφάλματα) είναι η απόκλιση, ως
προς το $y$, των δειγμάτων από την γραμμή παλινδρόμησης και βρίσκεται από τον τύπο:
$$c=y_i-\widehat{y}_i$$
Παίρνοντας τώρα το άθροισμα των τετραγώνων των σφαλμάτων (\textlatin{Sum of Squared Residuals}):
$$\text{\textlatin{SSR}}=\sum\limits_{i=1}^{n}e_i^2$$
Ψάχνουμε μία ευθεία γραμμή (παλινδρόμησης) η οποία θα ελαχιστοποιεί το άθροισμα. Στην
περίπτωση της απλής γραμμικής παλινδρόμησης το άθροισμα ελαχιστοποιείται για:
\begin{gather*}
    a=\frac{\sum(x_i-\widetilde{x})(y_i-\widetilde{y})}{\sum(x_i-\widetilde{x})^2} \\
    b=\widetilde{y}-a\widetilde{x}
\end{gather*}
Το τετράγωνο των σφαλμάτων επιλέχτηκε έτσι ώστε να μην αλληλοαναιρούνται τα θετικά με
τα αρνητικά σφάλματα. Το πλεονέκτημα έναντι της απόλυτης τιμής είναι ότι το τετράγωνο
είναι μία συνεχής και παραγωγίσιμη συνάρτηση. Για ποιο περίπλοκες μορφές γραμμικής
παλινδρόμησης χρησιμοποιούνται και επαναληπτικές μέθοδοι προσέγγισης της καλύτερης
ευθείας (\textlatin{line of best fit}), εξετάζοντας επαναληπτικά τιμές για την κλίση και την μετατόπιση
μέχρι να βρούμε το ζευγάρι που ελαχιστοποιεί το άθροισμα των τετραγώνων.

\sloppy
Μια μετρική της απλής γραμμικής παλινδρόμησης είναι το μέσο τετραγωνικό σφάλμα (\textlatin{Mean Square Error}),
όπου για μία σταθερή διακύμανση του σφάλματος το \textlatin{MSE} υπολογίζεται ως:
\fussy
$$\sigma_\epsilon^2=\frac{\text{\textlatin{SSR}}}{n-p-1}$$
όπου:
\begin{description}
    \item[$n$ :] πλήθος δειγμάτων
    \item[$p$ :] πλήθος ανεξάρτητων μεταβλητών (\textlatin{regressors})
\end{description}
Για την περίπτωση της απλής γραμμικής παλινδρόμησης που εξετάζουμε τώρα το $p=1$

Μια άλλη μετρική που χρησιμοποιείται κατά κόρον είναι το $R^2$ και στη συνέχεια θα δούμε πως υπολογίζεται.
Αρχικά υπολογίζουμε τον μέσο όρο των δειγμάτων (ως προς τον άξονα της εξαρτημένης
μεταβλητής). Η τιμή αυτή θα είναι το σημείο τομής του άξονα της εξαρτημένης μεταβλητής
και μίας ευθείας παράλληλης ως προς τον άξονα της ανεξάρτητης μεταβλητής. Αυτή η ευθεία
είναι μία ευθεία παλινδρόμησης (\textlatin{mean}) όταν δεν παίρνουμε υπόψιν μας την εξάρτηση από
την ανεξάρτητη μεταβλητή. Υπολογίζουμε το \textlatin{SSR} και την διακύμανση γύρο από αυτή την
γραμμή παλινδρόμησης. Έπειτα υπολογίζουμε την διακύμανση και γύρο από την γραμμή
παλινδρόμησης που εξετάζουμε (\textlatin{fit}). Το $R^2$ υπολογίζεται από τον τύπο:
$$R^2=\frac{\text{\textlatin{var}}(\text{\textlatin{mean}})-\text{\textlatin{var}}(\text{\textlatin{fit}})}{\text{\textlatin{var}}(\text{\textlatin{mean}})}$$

Η σχέση αυτή μας δείχνει την μείωση της διακύμανσης όταν παίρνουμε υπόψιν μας την
ανεξάρτητη μεταβλητή ή αλλιώς το ποσοστό της μεταβολής της εξαρτημένης μεταβλητής που
εξηγεί η μεταβολή της μεταβλητής που εξετάζουμε ή θα μπορούσαμε να πούμε ότι
προσδιορίζει το \textbf{ποσοστό της βεβαιότητας} που έχουμε όταν κάνουμε μια πρόβλεψη της τιμής
της εξαρτημένης για μία συγκεκριμένη τιμή της ανεξάρτητης. Το ίδιο αποτέλεσμα θα είχαμε
και εάν υπολογίζαμε το $R^2$ με τα \textlatin{SSR} αντί της διακύμανσης λόγω του ότι απαλείφονται οι
παρονομαστές.

Το $R^2$
είναι προφανές ότι παίρνει τιμές από $0$ ως $1$ και όσο το μεγαλύτερο
τόσο το καλύτερο. Είναι μια μετρική μεγάλης σημασίας για τον προσδιορισμό της
ακαταλληλότητας μιας ανεξάρτητης μεταβλητής, καθώς για μεγάλα δείγματα ο
προσδιορισμός αυτός και η απόρριψη μεταβλητών μικρότερης σημασίας οδηγεί σε μεγάλη
μείωση του χρόνου υπολογισμού και πόρων. Ωστόσο χρειαζόμαστε έναν τρόπο για να
προσδιορίζουμε το πόσο σημαντικό είναι στατιστικά (πόσο ακριβές είναι), καθώς όπως
φαίνεται και παρακάτω ένα με την προσθήκη έξτρα δεδομένων το $R^2$ αυξάνεται χωρίς να έχει
βελτιωθεί απαραίτητα το μοντέλο μας.

To \en{\textbf{p-value}} είναι μια μετρική που κάνει ακριβός αυτή την δουλειά. Το \textlatin{p-value} για το $R^2$
υπολογίζεται από έναν άλλον όρο το $F$. Το $F$ μοιάζει πολύ με το $R^2$, η μόνη διαφορά τους
βρίσκεται στον παρονομαστή με το $R^2$
να είναι:
$$R^2=\frac{\text{διακύμασνη του }y\text{ που εξηγείται από το }x}{\text{διακύμασνη του }y\text{ όταν δεν παίρνουμε υπόψιν το }x}$$
Ενώ το $F$:
$$F=\frac{\text{διακύμασνη του }y\text{ που εξηγείται από το }x}{\text{διακύμασνη του }y\text{ που δεν εξηγείται από το }x}$$
Ή αλλιώς:
$$\mathlarger{F =\frac{\enm{SS}(\enm{mean}) - \frac{\enm{SS}(\enm{fit})}{p_{\enm{fit}}} - p_{\enm{mean}}} {\frac{\enm{SS}(\enm{fit})}{n} - p_{\enm{fit}}}}$$
Όπου:
\begin{description}
    \item[$p_{\enm{fit}}$ :] παράμετροι (βαθμοί ελευθερίας) για την ευθεία που εξετάζουμε (2 για την απλή γραμμική παλινδρόμηση)
    \item[$p_{\enm{mean}}$ :]  παράμετροι (βαθμοί ελευθερίας) για την ευθεία αναφοράς = 1 (αφού η μόνη παράμετρος που εξετάζουμε είναι η μετατόπιση $b$)
    \item[$n$ :] αριθμός των δεδομένων
\end{description}

Ο υπολογισμός τώρα του \en{p-value} από το $F$ γίνεται μέσω της παραγωγής τυχαίων δεδομένων
και υπολογίζοντας το $F$ για αυτό το σετ. Αυτό γίνεται για αρκετές επαναλήψεις (όσο
περισσότερες τόσο το καλύτερο) και τα αποτελέσματα ($F$) που παίρνουμε τα τοποθετούμε σε
ένα ιστόγραμμα. Έπειτα βρίσκουμε το $F$ για το σετ δεδομένων που εξετάζουμε και το
τοποθετούμε και αυτό στο ιστόγραμμα.

Το p-value υπολογίζεται από τον αριθμό των τιμών
που είναι μεγαλύτερες από το $F$ του σετ που εξετάζουμε διαιρεμένο με τον αριθμό όλων των
τιμών. Στην πράξη το γράφημα του ιστογράμματος προσεγγίζεται από μία γραμμή $F$
κατανομής, έτσι ώστε να μην χρειαστεί να παράξουμε τόσο μεγάλο όγκο τυχαίων δεδομένων.
Το σχήμα της γραμμής εξαρτάται από τον αριθμό των δεδομένων και τους βαθμούς
ελευθερίας. Το \en{p-value} παίρνει μικρότερες τιμές όσο περισσότερα δείγματα έχουμε ανά
παράμετρο. Όσο μικρότερο το \en{p-value} τόσο καλύτερο το $R^2$
και κατά συνέπεια η γραμμή
παλινδρόμησης