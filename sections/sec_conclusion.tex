\newpage
\section{Συμπέρασμα}
Σε αυτή την εργασία είδαμε τα διαφορετικά προβλήματα που λύνει η επιστήμη των δεδομένων (ταξινόμηση, παλινδρόμηση και ομαδοποίηση). Επιπλέον είδαμε πολλές διαφορετικές μεθόδους που μπορούμε να προσεγγίσουμε κάθε ένα από αυτά τα προβλήματα.
Είδαμε μεθόδους οι οποίες οποίες απαιτούσαν από τον χρήστη να ορίσει κάποιες παραμέτρους ή κάποιες που ήταν χρήσιμες μόνο σε συγκεκριμένες περιπτώσεις.

Τελικά καταλήξαμε ότι κυρίαρχος τρόπος να λυθούν αυτά τα προβλήματα είναι τα νευρωνικά δίκτυα. Τα νευρωνικά δίκτυα είναι μία πολύ γενικευμένη μέθοδος και υπάρχουν πολλά είδη για κάθε είδος προβλήματος που αναλύσαμε και πολύ περισσότερα ακόμα. Τα νευρωνικά δίκτυα έχουν την ικανότητα μάθησης το οποία τα καθιστά πολύ βολικά για κάθε πρόβλημα.