\newpage
\section{Νευρωνικά Δίκτυα}
Τα νευρωνικά δίκτυα είναι στις μέρες μας το κυρίαρχο μοντέλο στη μηχανική μάθηση και μπορεί να λύσει και τους τρεις τύπους προβλημάτων που έχουμε δει (ταξινόμηση, παλινδρόμηση και ομαδοποίηση). Τα τεχνητά νευρωνικά δίκτυα δημιουργήθηκαν με
σκοπό να μιμηθούν τον τρόπο που λειτουργεί ο εγκέφαλος του ανθρώπου. Οι νευρώνες του εγκεφάλου μας λειτουργούν δημιουργώντας συνδέσεις με άλλους νευρώνες και όλοι μαζί συμβάλουν στην επεξεργασία ενός ηλεκτρικού σήματος που έρχεται από τον
εγκέφαλο το οποίο τελικά καταλήγει να είναι απλή καθημερινή πράξη για τον άνθρωπο (όπως για παράδειγμα να κουνήσει το χέρι του)\cite{nnip}.

Τα τεχνητά νευρωνικά δίκτυα προσπαθούν λοιπόν να αντιγράψουν αυτη ακριβώς την ιδιότητα. Ο σκοπός τους είναι να παίρνουν μια είσοδο και να παράγουν μία απάντηση. Για παράδειγμα θα μπορούσαμε σαν είσοδο να δώσουμε τις αιματολογικές εξετάσεις
ενός ανθρώπου και η έξοδος να είναι 0 ή 1 ανάλογα αν έχει μια ασθένεια ή όχι. Αυτό είναι ένα πρόβλημα δυαδικής ταξινόμησης αλλά θα μπορούσαμε να λύσουμε και πολλά άλλα προβλήματα και θα αναλύσουμε στην συνέχεια τις αλλαγές που πρέπει να
κάνουμε σε ένα νευρωνικό δίκτυο ανάλογα το πρόβλημα. Σε αυτή την ενότητα θα αναλύσουμε διάφορες έννοιες τις οποίες πρέπει να γνωρίζει κανείς αν θέλει να κατανοήσει τα νευρωνικά δίκτυα. Αυτές είναι\cite{nnav}:
\begin{itemize}
    \item Νευρώνας
    \item Επίπεδο
    \item Οπισθοδρόμηση (\en{back propagation})
    \item Συναρτήσεις ενεργοποίησης
    \item Συναρτήσεις σφάλματος
\end{itemize}

\subsection{Νευρώνας}
\subsection{Επίπεδο}
\subsection{Οπισθοδρόμηση}
\subsection{Συναρτήσεις ενεργοποίησης}
\subsection{Συναρτήσεις σφάλματος}