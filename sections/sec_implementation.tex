\newpage
\section{Υλοποίηση}
Όπως έχουμε ήδη αναφέρει, για την υλοποίηση αλγορίθμων θα χρησιμοποιήσουμε την γλώσσα \textlatin{C++} και σκοπός μας είναι τα μοντέλα που θα δημιουργήσουμε να είναι εύχρηστα και επαναχρησιμοποιήσιμα και από
άλλους χρήστες. Θα χρειαστεί επομένως να δημιουργήσουμε κάποια εργαλεία με γενική χρήση.
\subsection{\textlatin{CSV Parser}}
Πριν μπορέσουμε να ξεκινήσουμε την υλοποίηση αλγορίθμων θα χρειαστούμε έναν τρόπο για να μεταφέρουμε τα δεδομένα μας μέσα στο πρόγραμμα μας. Ο πιο συνηθισμένος τύπος αρχείου για δεδομένα είναι αυτός με την
κατάληξη \textlatin{.csv (comma separated values)}. Όπως φαίνεται και από το όνομα είναι ένα αρχείο στο οποία τα δεδομένα είναι διαχωρισμένα με κόμμα. Ένα παράδειγμα είναι το παρακάτω:
\begin{otherlanguage}{english}
\begin{lstlisting}[style=csvstyle, caption = \textlatin{diabetes.csv}]
glucose,bloodpressure,diabetes
40,85,0
40,92,0
45,63,1
45,80,0
40,73,1
45,82,0
40,85,0
30,63,1
65,65,1
45,82,0
\end{lstlisting}
\end{otherlanguage}
Το παραπάνω αρχείο μας δίνει δείγματα τα οποία έχουν ως στοιχεία τη γλυκόζη, την πίεση και τα αν έχουν ή όχι διαβήτη, όπου κάθε καινούρια σειρά είναι και ένα διαφορετικό δείγμα. Θα φτιάξουμε μια συνάρτηση τώρα ώστε να εισάγουμε αυτά τα
δεδομένα στο πρόγραμμα μας. Επιπλέον θα φτιάξουμε και μια συνάρτηση για να εμφανίζουμε αυτά τα δεδομένα ώστε να επιβεβαιώσουμε ότι η συνάρτηση δουλεύει σωστά. Για να το κάνουμε αυτό θα χρησιμοποιήσουμε κάποια \textlatin{templates} που μας
παρέχει η \textlatin{C++} τα οποία υπάρχουν στους παρακάτω \textlatin{headers}:
\begin{otherlanguage}{english}
\begin{lstlisting}[style=cppstyle, caption = parser includes]
#pragma once

#include <iostream>
#include <iomanip>
#include <fstream>
#include <vector>
#include <unordered_map>
\end{lstlisting}
\end{otherlanguage}
Θα χρησιμοποιήσουμε τα πρώτα δύο \textlatin{includes} για να εμφανίσουμε τα δεδομένα μορφοποιημένα στο \textlatin{command line}, ενώ τα επόμενα τρία θα τα χρειαστούμε για να εισάγουμε τα δεδομένα.
\begin{otherlanguage}{english}
\begin{lstlisting}[style=cppstyle, caption = parse\_csv function]
template <class T>
std::unordered_map<std::string, std::vector<T>> parse_csv (
    std::string filepath, size_t features)
{
    // The data will be held in an unordered map
    std::unordered_map<std::string, std::vector<float>> data, tempData;
    // Csv file to read from
    std::ifstream file;
    T temp;
    std::string key;
    file.open(filepath);
    for (size_t i = 0; i < features - 1; i++) {
        std::getline(file, key, ',');
        tempData.insert({key, {}});
    }
    std::getline(file, key, '\n');
    tempData.insert({key, {}});
    for (const auto &pair : tempData) data.insert(pair);
    bool eof = false;
    while (true) {
        for (auto &pair : data) {
            if (file >> temp) {
                pair.second.push_back(temp);
                if (    file.peek() == ','  ||
                        file.peek() == '\n' ||
                        file.peek() == '\r') file.ignore();
            }
            else {
                eof = true;
                break;
            }
        }
        if (eof) break;
    }
    file.close();
    return data;
}
\end{lstlisting}
\end{otherlanguage}
\par Το \textlatin{map} είναι μια δομή δεδομένων η οποία αντιστοιχίζει έναν τύπο μεταβλητης σε μία άλλη. Στην συγκεκριμένη περίπτωση αντιστοιχίζουμε \textlatin{strings} και συγκεκριμένα τα ονόματα των χαρακτηριστικών των δεδομένων μας σε
\textlatin{vectors} δηλαδή πίνακες. Μπορούμε να χρησιμοποιήσουμε  τώρα το \textlatin{data} για να πάρουμε όποια τιμή θέλουμε. Για παράδειγμα το \textlatin{data["glucose"]} θα μας επιστρέψει έναν πίνακα που θα περιέχει όλη την πρώτη στήλη
από το \textlatin{diabetes.csv} και αν θέλουμε να πάρουμε ένα συγκεκριμένο στοιχείο από αυτή τη στήλη μπορούμε απλά να το πάρουμε γνωρίζοντας τη θέση του. Το \textlatin{data["glucose"][2]} θα μας επιστρέψει το τρίτο στοιχείο της στήλης.
\par Τωρα θα ορίσουμε τη συνάρτηση που θα εμφανίζει τα αποτελέσματα στην οθόνη.
\begin{otherlanguage}{english}
\begin{lstlisting}[style=cppstyle, caption = print\_map function]
template <class T>
void print_map (
    const std::unordered_map<std::string,std::vector<T>>& data)
{
    size_t size;
    for (const auto &pair : data) {
        std::cout << pair.first << '\t';
        size = pair.second.size();
    }
    std::cout << '\n';
    for (size_t i = 0; i < size; i++) {
        for (const auto &pair : data)
            std::cout   << std::setw(pair.first.size())
                        << std::left
                        << pair.second[i]
                        << '\t';
        std::cout << '\n';
    }
}
\end{lstlisting}
\end{otherlanguage}
Για το παράδειγμά μας θα πάρουμε την εξής έξοδο:
\begin{otherlanguage}{english}
\begin{lstlisting}[style=csvstyle, caption = output for diabetes.csv]
glucose bloodpressure   diabetes
40      85              0
40      92              0
45      63              1
45      80              0
40      73              1
45      82              0
40      85              0
30      63              1
65      65              1
45      82              0
\end{lstlisting}
\end{otherlanguage}
%include dont forget